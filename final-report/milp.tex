\section{MILP Method using Monolingual Corpus}

This this chapter we reproduce the state of the art in adjective ranking by Bansal and de Melo \newcite{demelo:13}. 

\subsection{Problem Formulation}

Bansal and de Melo use pairwise co-occurrence of adjective pairs around the paraphrases found in table 1 to infer the relative strength between the adjectives. Because this data is missing for most pairs, pairwise rankings are computed when possible, otherwise Bansal and de Melo used the transitive property of partial rankings to infer the missing relationships. Pairwise ranking is computed as:

\[ score(a_1, a_2) = \frac{(W_1 - S_1) - (W_2 - S_2)}{cnt(a_1) \cdot cnt(a_2)}, \]

where:

\[ W_1 = \frac{1}{P_1} \sum_{p \in P_{ws}} cnt(p(a_1, a_2))\]

\[ W_2 = \frac{1}{P_1} \sum_{p \in P_{ws}} cnt(p(a_2, a_1))\]

\[ S_1 = \frac{1}{P_2} \sum_{p \in P_{sw}} cnt(p(a_1, a_2))\]

\[ S_2 = \frac{1}{P_2} \sum_{p \in P_{sw}} cnt(p(a_2, a_1)),\]

with:

\[ P_1 = \sum_{p \in P_{ws}} cnt(p(*, *))\]

\[ P_2 = \sum_{p \in P_{sw}} cnt(p(*, *)).\]


Observe $W_1$ measures the likelihood of encountering the phrase $a_1 p a_2$ conditioned on the fact that the corpus is composed 
entirely of phrases of form $* p *$; a similar interpretation holds for $W_2$, $S_1$, and $S_2$. Furthermore, $(W_1 - S_1) - (W_2 - S_2)$ is positive when $a_1$ occurs more often on the weaker side of the intensity scale relative to $a_2$, hence $score(a_1, a_2)$ is an $cardinal$ measure how much weaker $a_1$ is relative to $a_2$. The denominator $cnt(a_1) \cdot cnt(a_2)$ penalizes high absolute value of the numerator due to higher frequency of certain words, thus normalizing the score over all pair of adjectives, and therefore global comparison is well defined over some cardinal scale. Finally, observe that $score(a_1, a_2) = - score(a_1, a_2)$.

Given pairwise scores over a cluster of adjectives where a global ranking is known to exist, Bansal then aim to recover the ranking using mixed integer linear programming. Assuming we are given $N$ input words $A = \{a_1, ..., a_N\}$, the MILP formulation places them on a scale $[0,1]$ by assigning each $a_i$ a value $x_i \in [0,1]$. The objective function is formulated so that if $score(a_i, a_j)$ is greater than zero, then we know $a_i$ is weaker than $a_j$ and the optimal solution should have $x_i < x_j$. The entire formulation is reproduced below:

{\bf Maximize}

\[ \sum_{i,j} (w_{ij} - s_{ij}) \cdot score(a_i, a_j)\]

{\bf s.t}

\begin{align*}
  d_{ij} &= x_j - x_i          &\forall i,j \in \{1,...,N\}\\
  d_{ij} &- w_{ij}C \leq 0     &\forall i,j \in \{1,...,N\}\\
  d_{ij} &+ (1 - w_{ij})C > 0  &\quad \forall i,j \in \{1,...,N\}\\
  d_{ij} &+ s_{ij}C \geq 0     &\forall i,j \in \{1,...,N\}\\
  d_{ij} &- (1 - s_{ij})C < 0  &\forall i,j \in \{1,...,N\}\\
  x_i    &\in [0,1]            &\forall i,j \in \{1,...,N\}\\
  w_{ij} &\in \{0,1\}          &\forall i,j \in \{1,...,N\}\\
  s_{ij} &\in \{0,1\}          &\forall i,j \in \{1,...,N\}.
\end{align*}

Note $d_{ij}$ captures the difference between $x_i$ and $x_j$, $C$ is a very large constant greater than $\sum_{i,j} |score(a_i,a_j)|$. If the variable $w_{ij} = 1$, then we conclude $a_i < a_j$, and vice versa for $s_{ij}$. The objective function encourages $w_{ij} = 1$  for $score(a_i,a_j) > 0$ and $w_{ij} = 0$ otherwise. Furthermore, note either $s_{ij}$ or $w_{ij}$ can be one, thus the optimal solution does not have ties. Bansal then extended the objective to incorporate synonymy information over the $N$ adjectives, defined by $E \subseteq \{1,...,N\} \times \{1,...,N\}$. The objective is now to maximize:

\[ \sum_{(i,j) \not\in E} (w_{ij} - s_{ij}) \cdot score(a_i, a_j) - \sum_{(i,j) \in E} (w_{ij} + s_{ij}) \cdot C,\]

while the constraints remain unchanged. The additional set of terms encourages both $s_{ij}$ and $w_{ij}$ to be zero if both $a_i$ and $a_j$ are in $E$, thus the optimal solution may contain synonyms. 


\subsection{Results}

In this section we report both the results of Bansal's method as stated in their paper, and the results we reproduced. Table 3 reports all results from the different approaches. Note that our reproduction of Bansal's MILP is accurate with acceptable errors. However, we were not able to reproduce MILP with synonymy accurately because Bansal relied on synonyms marked by annotators, these annotations were not released in the public code base accompanying the paper. Instead we attempted to replicate the experiment using synonyms used by word-net, and observed a marked decrease in accuracy across all measures. However since the synonym set are presumably different, these results cannot be meaningful and thus are not reported. 

\begin{table}
\small
\centering
\begin{tabular}{|l|c|c|c|c|c|}
% 
\hline 
\bf Method & \bf Pairwise Accuracy & \bf Avg. $\tau$ & \bf Avg. $|\tau|$ & \bf Avg. $\tau'$ & \bf Avg. $|\tau'|$ \\ 
\hline
% 
Inter-Annotator Agreement           & 78.0\% & 0.67 & 0.76 & N/A & N/A         \\
Gold Standard                       & 100.0\% & 0.90 & 0.90 & 0.90 & 0.90         \\
\hline
% 
% 
MILP reported                       & 69.6\% & 0.57 & 0.65 & N/A & N/A         \\
MILP with synonymy reported         & 78.2\% & 0.57 & 0.66 & N/A & N/A         \\
\hline
% 
MILP reproduced                     & 68.0\% & 0.55 & 0.64 & 0.41 & 0.54         \\
MILP with new patterns reproduced   & 69.6\% & 0.57 & 0.66 & 0.44 & 0.56 \\
\hline
\end{tabular}
\caption{\label{font-table} Main results using Bansal's patterns. Note $\tau$ refers to kendall's $\tau$ with ties, while $\tau'$ referrs to the variant where ties are not considered.}
\end{table}\newpage

